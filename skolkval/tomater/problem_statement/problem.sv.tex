\problemname{Tomater}
Ett intressant faktum är att omogna tomater mognar snabbare om man lägger in några redan mogna tomater bland dem.
I denna uppgift ska du simulera denna process och räkna ut hur många tomater som är mogna efter en viss tid.

Antag att $n$ tomater ligger i en lång rad och är numrerade från $1$ till $n$.
Tre av dessa tomater, nummer $t_1$, $t_2$ och $t_3$, är redan mogna när simuleringen startar vid dag $0$.
Varje dag mognar de tomater som ligger precis intill en redan mogen tomat.
Efter dag $1$ har alltså grannarna till de tre första mogna tomaterna mognat, efter dag $2$ har även grannarna till de som mognade under dag $1$ mognat och så vidare.

Skriv ett program som givet antal tomater $n$, antalet dagar $d$, och numren $t_1$, $t_2$, $t_3$, beräknar hur många tomater som är mogna efter $d$ dagar.

\section*{Indata}
På första raden av indata står de två talen $n$ ($3 \le n \le 100$) och $d$ ($1 \le d \le 100$).

På den andra raden står numren $t_1$, $t_2$ och $t_3$, alla olika och i intervallet $1 \dots n$.

\section*{Utdata}
Skriv ut ett enda tal: antalet mogna tomater efter $d$ dagar.
